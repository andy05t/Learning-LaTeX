\documentclass{article}

% A way to import packages: \usepackage{PACKAGENAME}
% Provides a full package \usepackage{textlive-full}
\usepackage{amsmath}
\usepackage{comment}
\usepackage{graphicx}
\usepackage{subcaption}
\usepackage{fancyhdr}
\usepackage{hyperref} %adding a hyperlink

\pagestyle{fancy}
\lhead{Header} %left header
\rhead{This is page \thepage} %right header
\cfoot{Footer!} %center footer
\renewcommand{\headrulewidth}{0.4pt}
\renewcommand{\footrulewidth}{0.4pt}

\title{About me - For Google Code-in}
\date{January 2020}
\author{Andrey T}

\begin{document}

\maketitle
 
\centering My first \LaTeX{} document. Welcome!

\pagenumbering{gobble}
\newpage
\pagenumbering{arabic} %Not necessarily needed - they are already in arabic, you 	can also have 'roman' or 'gobble', where gobble removes numbering from the page

\tableofcontents  %Adds a fabuluous table of contents - just so impressed!
\newpage

\section{Section 1: About Me}
\textbf{About me:} Hello, my name is Andrey T. and I love programming!\\
\vspace{10px}
I fell in love with \LaTeX{} from the first I started using it. This is the first time I am participating in Google Code-in and I am finding it very exciting. There are many different tasks which you can do, and they have been very helpful to develop my knowledge and expand it further.

\subsection{Experimenting: Subsection - Maths}

Just to write a simple equation $f(x) = x^2 - x$ surround it with dollar signs, 
or use an environment:

\begin{equation*}
  1 + 2 = 3 
\end{equation*}

\begin{equation*}
  1 = 3 - 2
\end{equation*}

\begin{align*}
  1 + 2 &= 3\\
  1 &= 3 - 2
\end{align*}
Align environment aligns the equations at the = sign.\\ %\\ represents an enter
 Other more complex symbols include:   %\newline is pretty much \\

\begin{align*}
  f(x) &= x^2\\
  g(x) &= \frac{1}{x}\\
  F(x) &= \int^a_b \frac{1}{3}x^3\\
  \frac{1}{\sqrt{x}}
\end{align*}



\subsubsection{Subsubsection}

Just experimenting...

\paragraph{Line experimenting}
This is an example of a multi-line comment

\begin{comment}
\\ start a new paragraph.
\- OK to hyphenate a word here.
\cleardoublepage flush all material and start a new page, start new odd numbered page.
\clearpage plush all material and start a new page.
\hyphenation enter a sequence pf exceptional hyphenations.
\linebreak allow to break the line here.
\newline request a new line.
\newpage request a new page.
\nolinebreak no line break should happen here.
\nopagebreak no page break should happen here.
\pagebreak encourage page break.
\end{comment}

\subparagraph{Subparagraph}

Even more text...

\newpage %Creates a new page

\section{Section 2: Images}

\begin{figure}[h!]
  \includegraphics[width=\linewidth]{/home/andrey/Documents/HTML/example-image.jpg} %Provide a full path to file
   %linewidth = wide as the page
  \caption{A fish} %Caption under picture
  \label{fig:fish1} %Not seen = for further reference
\end{figure}

Figure \ref{fig:fish1} shows an agile fish.\\ %\ref is a reference to the name of the picture

Possible values for reference: 
h (here) - same location
t (top) - top of page
b (bottom) - bottom of page
p (page) - on an extra page
! (override) - will force the specified location

\begin{figure}[h!]
  \centering
  \begin{subfigure}[b]{0.4\linewidth}
    \includegraphics[width=\linewidth]{/home/andrey/Documents/HTML/example-image.jpg}
    \caption{Fish}
  \end{subfigure}
  \begin{subfigure}[b]{0.4\linewidth}
    \includegraphics[width=\linewidth]{/home/andrey/Documents/HTML/example-image.jpg}
    \caption{More Fish}
  \end{subfigure}
  \caption{The same fish. Twice.}
  \label{fig:fish}
\end{figure}

Just like this you can have many images, which os especially useful for comparing graphs, just don't forget the subcaption package. REMEMBER: Always set your images to 0.1 less than you expect, in order for them to fit in.

\section{Section 3: Text formatting}
Some of the \textbf{greatest}
discoveries in \underline{science} 
were made by \textbf{\textit{accident}}.
\newline
So, \textbf{textbf} makes text bold, \underline{underlines} underlines and \textit{italicises} italicises.\\

This is some example text\footnote{\label{myfootnote}Just explaining the words - just wow! I can write a book in this...}.

\subsection{Subsection: Hyperlinks!}
Remember to add \textbf{usepackage {hyperref}} before usage\\
This is a link: \href{https://www.latex-project.org/}{LaTeX-Project Home Page}.\\
You can also link to bare URLs without an additional description:\\ \url{https://www.latex-project.org/}\\

\vspace{10px}

\textbf{The most useful links I have used to create this document are:} \\ 
\url{https://www.latex-tutorial.com/tutorials/hyperlinks/}\\
\url{https://drive.google.com/file/d/1pkcCjcAjfiJWyFcWpvHuCGHnkz9mm-SE/view}\\
\url{https://www.overleaf.com/learn/latex/Learn_LaTeX_in_30_minutes#Downloading_your_finished_document}

\end{document}